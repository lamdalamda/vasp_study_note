\documentclass{article}
\usepackage[UTF8]{ctex}
\title{Thermodynamics}
\author{lamda}
\begin{document}

\section{Attention注意}

每次渲染index.html之后,在<head>部分把原本的stylesheet部分换成以下代码


<link rel="stylesheet" href="https://cdn.jsdelivr.net/npm/katex@0.10.2/dist/katex.min.css" integrity="sha384-yFRtMMDnQtDRO8rLpMIKrtPCD5jdktao2TV19YiZYWMDkUR5GQZR/NOVTdquEx1j" crossorigin="anonymous">
<script defer src="https://cdn.jsdelivr.net/npm/katex@0.10.2/dist/katex.min.js" integrity="sha384-9Nhn55MVVN0/4OFx7EE5kpFBPsEMZxKTCnA+4fqDmg12eCTqGi6+BB2LjY8brQxJ" crossorigin="anonymous"></script>
<script defer src="https://cdn.jsdelivr.net/npm/katex@0.10.2/dist/contrib/auto-render.min.js" integrity="sha384-kWPLUVMOks5AQFrykwIup5lo0m3iMkkHrD0uJ4H5cjeGihAutqP0yW0J6dpFiVkI" crossorigin="anonymous" onload="renderMathInElement(document.body);"></script>


https://katex.org/docs/autorender.html

\section{DFT Theory}
 latex test docs index


\subsection {basic concepts 基本概念:如何计算}
摘自David S. Sholl, _Density functional Theory_

\subsubsection {Hohenberg-Kohn 1st Theorem :}

The ground-state energy from Schroginger's equation is a unique functional of the electron density.

OR

ground state electron density uniquely determines all properties: energy/ wavefunction.

基态能量是电子密度分布函数(或者说,基态电子在空间中的密度分布)的泛函
一个基态电子密度对应了唯一的能量、波函数。

问题:泛函形式未知

\subsubsection {Hohenberg-Kohn 2nd Theorem :}

The electron density that minimizes the energy of the overall functional is the true electron density corresponding to the full solution of the schrodinger equation.

使这个泛函结果(能量)最低的电子密度是薛定谔方程的解

->如果知道了泛函的形式,那么就可以计算出基态电子密度

\subsubsection {n(r), 或者基态电子密度}

基态电子密度 $n_{(r)} = 2\Sigma \phi_i^*(r)\phi_i(r)$

更清楚一点

$$n_{(x,y,z)} = 2\phi_1^*(x,y,z)\phi_1(x,y,z)+2\phi_2^*(x,y,z)\phi_2(x,y,z)+2\phi_3^*(x,y,z)\phi_3(x,y,z)+.....$$

缩成一维:

$n_{(x)} = 2\phi_1^*(x)\phi_1(x)+2\phi_2^*(x)\phi_2(x)+2\phi_3^*(x)\phi_3(x)+.....$,对所有电子

是个期望值一样的东西

\subsubsection {泛函的具体形式 the energy functional=?}

对于单个电子已知如下:

$E[\phi_i(r)]=E_{known}[\phi_i(r)]+E_{unknown}[\phi_i(r)]$

方括号是泛函的意思。类似于函数,把圆括号变成方括号
$ n(r) -- E[] --> E[n(r)] $


$E_{known}[]$是已知的能量泛函,包括电子动能,电子-核与电子-电子

$E_{unknown}[]$是未知的能量泛函:交换能


其中

$E_{known}[\phi_i(r)]=\frac{-\hbar}{m}\Sigma\int\phi_i^*(r)\nabla^2\phi_i(r)d^3r+\int V(r)n(r)d^3r+e^2/2\int \int \frac {n(r)n(r')}{|r-r'|}d^3rd^3r'+E_{ion} $

缩成一维,对于0号电子$\phi_0$

$E_{known}[\phi_0(x)]=\frac{-\hbar}{m}\Sigma\int\phi_0^*(x)\phi_i''(x)dx+\int V(x)n(x)dx+e^2/2\int \int \frac {n(x)n(x')}{|x-x'|}dxdx'+E_{ion} $

(不太清楚为什么会有一个求和)

由于前述$n_{(r)} = 2\Sigma \phi_i^*(r)\phi_i(r)$

所以虽然不知道是否合理,但是推导出来这个公式,即phi和n(r)有同样的泛函??

$E[2\Sigma \phi_i^*(r)\phi_i(r)]=E_{known}[2\Sigma \phi_i^*(r)\phi_i(r)]+E_{unknown}[2\Sigma \phi_i^*(r)\phi_i(r)]$

也就是说泛函的形式是 
$E[]=E_{known}[]+E_{unknown}[]$

问题:解不出来电子密度n(r)因此需要kohn sham 方程解n(r)

\subsubsection {Kohn-Sham equation}

Kohn-Sham equation是用来找电子密度,即n(r):

Kohn-Sham 方程是:

$[\frac{-\hbar^2}{2m}\nabla^2+V(r)+V_H(r)+V_{XC}(r)]\phi_i(r)=\epsilon_i\phi_i(r)$

其中

$V(r)$是薛定谔方程中已知的,是电子和所有原子核的相互作用

$V_H(r)$是hartree potential,是 coulomb repulsion  between "this electron" and "the total electron density", 或者说电子i和整个电子密度n(r)之间的coulomb repulsion

$V_H(r)=e^2\int \frac{n(r')}{|r-r'|}d^3r'$
理解为以r点的坐标向外积分?

$V_{XC}$是未知的,虽然有
$V_{XC}=\frac{\delta E_{XC}(r)}{\delta n(r)} $

用人能看懂的方法重新写

对于0号电子,其具有能量$\epsilon_0$,缩到一维,波函数满足

$[\frac{-\hbar^2}{2m}\phi_0''(x)+V(x)+e^2\int \frac{n(x')}{|x-x'|}dx'+V_{XC}(x)]\phi_0(x)=\epsilon_0\phi_0(x)$

\subsubsection {递归求解}

1\猜测一个 初始电子密度

guess a initial trial electron density n(r)

2\ 解kohn sham 方程,得到每一个电子的波函数

solve the kohn-sham equation with n(r), get the $\phi(r)$

3\ 计算新的电子密度

通过$n_{(r)} = 2\Sigma \phi_i^*(r)\phi_i(r)$

4\ 调整电子密度递归运算

\subsubsection {结论:}

通过 Hohenberg-Kohn定理和Kohn-Sham方程,距离比较精确地求解薛定谔方程目前只差一个合适的$E_{XC}$泛函

\subsection{寻找合适的$E_{XC}$泛函}

\subsubsection {LDA近似:计算Exc的一种方式}

如前文所讲Exc泛函是很难知道形式的。

Exc只在一种情况下可以知道形式:当n(r)=常数,即uniform electron gas

$V_{XC}(r)=V^{electron gas}_{XC}[n(r)]$

*防止混淆,左边是函数$V_{XC}()$,右边是泛函$V^{electron gas}_{XC}[]$

使用LDA近似后,可以精确地解薛定谔方程,但是这个解不是真实解,因为薛定谔方程中的Exc是假的

\subsubsection {GGA}

Use local electron density and the local gradient in the electron density calculation

局部电子密度,和局部电子密度的梯度

主流有:PW91 和 PBE

\subsection {寻找合适的波函数-hartree fock}

泛函在前文已经基本找好了,这里介绍找波函数的过程

当不考虑Exc时候,可以对KohnSham方程进行比较好的计算:

\subsubsection {单个电子与多电子的波函数}

对于单个电子,其波函数是localized的,可以用类似于$y=e^{-x^2}$描述,没有周期性,且在远离x=0的地方y约为0。但是对于晶体等多电子体系,应该使用周期性的波函数,比如$y=sin^2(x)$

\subsubsection {Hartree product}
当总共只有一个电子时候,

$h\Chi=E\Chi$

会得到许多个解 $\Chi_j$
对于每个解$\Chi_1,\Chi_2,\Chi_3,....$
(假设$\Chi_1$对应了最低能量)
对应的能量分别是 $E_1,E_2,E_3,....$,

其中$E_1$是能量最低的基态

对于多个电子体系来说

当忽略了电子-电子相互作用时

Hamiltonian: $H=h_1+h_2+h_3+...$

$h_i$是第i个电子的动能和势能算符

可以得到
$\phi(x_1,x_2,....x_n)=\prod{\Chi_{ji}}$
其中j代表了能级(像上面j=1是基态)

对于基态来讲
$\phi(x_1,x_2,....x_n)=\Chi_{11}(x_1)\Chi_{12}(x_2)\Chi_{13}(x_3)...$

能量则是简单相加
$E=E_{11},E_{12},E_{13},....$,

以上方法求得的波函数由于忽略了电子-电子相互作用,所以存在以下问题:

_由于电子是费米子,所以当两个电子交换位置时候符号必须改变_

然而,当电子交换位置时候,按照上式,波函数的正负号没有变化

所以:引入Slater determinant

双电子的Slater determinant:

$$\phi(x_1,x_2)=\frac{1}{\sqrt2}det\left\{\begin{matrix}
\Chi_j(x_1)&\Chi_j(x_2)\\
\Chi_k(x_1)&\Chi_k(x_2)\\
\end{matrix}\right\}$$

$$\phi(x_1,x_2)=\frac{1}{\sqrt2}[\Chi_j(x_1)\Chi_k(x_2)-\Chi_k(x_1)\Chi_j(x_2)]$$

这个波函数满足pauli exclusiong principle:不区分电子, 当电子具有相同坐标时等于0

\subsubsection {hartree fock calculation }

hartree fock calculation 基于这个假设;原子核位置固定


Kohn sham 方程是
$[\frac{-\hbar^2}{2m}\nabla^2+V(r)+V_H(r)+V_{XC}(r)]\phi_i(r)=\epsilon_i\phi_i(r)$

而对于**单个**电子,HF calculation是: 
$[\frac{-\hbar^2}{2m}\nabla^2+V(r)+V_H(r)]\phi_i(x)=\E_j\Chi_j(x)$
相比于kohn sham减少了一个Exc项

由前面slater行列式可以看到,如果想要描述一个N个电子体系,那么需要知道$\Chi_1$到$\Chi_N$

对此,可以找到有限个(K个)函数,即$\phi_{1}(x)$到$\phi_{K}(x)$,使得这K个函数可以加权相加得到$\Chi_1$到$\Chi_N$
这K个函数被称为basis set
即:

$\Chi_j(x)=\sum^K_{i=1}\alpha_{j,i}\phi_{i}(x)$

写的清楚一点,对第一个$\Chi_1(x)$:

$\Chi_1(x)=\alpha_{1,1}\phi_{1}(x)+\alpha_{1,2}\phi_{2}(x)+...+\alpha_{1,K}\phi_{K}(x)$

现在已知$\phi_{1}(x)$到$\phi_{K}(x)$,问题变成了找到$\alpha_{j,i}$

找到$\alpha_{i,j}$的方式:继续进行递归运算:

1, 猜测一组初始的$\alpha_{j,i}$

2,通过这个初始的$\alpha_{j,i}$计算出来一组n(r)

3, n(r)代回$\Chi_j(x)=\sum^K_{i=1}\alpha_{j,i}\phi_{i}(x)$方程,求解$\alpha_{j,i}$

通过这个递归运算,可以得到一个精确的能量值,这个能量值是Hartree fock limit.

这个值虽然精确,但并不是实际的电子能量, 因为没有考虑到Exc,即Hartree fock方法实际上并没有解决电子-电子相互作用

\subsubsection{DFT计算过程总结}

结合上文Kohn sham 方程计算,完整的DFT计算过程如下:

1, 猜测一组初始的$\alpha_{j,i}$

2,通过这个初始的$\alpha_{j,i}$计算出来一组n(r)

3,通过这个n(r) 解kohn sham方程 (或者不精确的Kohn sham方程,这个时候计算的是hartree fock limitx),得到每一个电子的波函数

4, 用每一个电子波函数计算n(r)

(通过$n_{(r)} = 2\Sigma \phi_i^*(r)\phi_i(r)$)

5,将n(r)代回$\Chi_j(x)=\sum^K_{i=1}\alpha_{j,i}\phi_{i}(x)$方程,求解$\alpha_{j,i}$

6,得到的$\alpha_{j,i}$代回第二步

\subsubsection {Exc}

Electron correlation energy = True system energy - Hartree fock limit.

即Exc就是实际的能量减去Hartree fock limit


\subsection {一些补充}
参考书:Richard M. Martin _Electronic structure: Basic theory and practical method_

\subsection {物质的性质: property of matters}

物质属性一般会取决于两个种类:
1、electronic ground state 基态电子
2、electronic excited state 激发态电子

原因是;原子核质量远大于电子,因此相对于电子来说可以基本看作静止
这就是Born-Oppenheimer approximation

电子最低能量态决定了原子核的结构
the lowest energy state of the elctrons determines the spatial structure of nuclei:

\subsection {bonding:}
一般认为分为5类:
closed shell system:稀有气体
ionic
metallic
covalent
hydrogen:比较特殊,因为氢原子是唯一没有core electron的 (往上面的氦的core electron是1s的两个电子)

实际材料的bonding一般是这五种的结合

\subsection {Hartree atomic units}

$\hbar =m_e = e = 4\pi /\epsilon_0 = 1$

定义以上单位为1,用来简化计算

# 晶体学知识 crystallography
参考书:Richard M. Martin _Electronic structure: Basic theory and practical method_

A crystal can be completely specified by the types and positions of the nuclei in **one repeat unit (primitive unit cell)** annd the rules that describe the **repetition (translation)**

primitive cell有很多种,wigner seitz cell是其中一种

Wigner Seitz cell: the most compact cell that is symmetric about the origin **注意 wigner seitz cell是在real space/实空间** 

wigner seitz cell在倒空间对应的是**first brillouin zone**

_fcc 的 wigner seitz cell是bcc的first brillouin zone_



\subsection {fourier transformation}

![Image](/Fourier_series_and_transform.gif)
![Image](/40cf849e55ed95732a60b52d4019d609_720w.jpg)
_https://zhuanlan.zhihu.com/p/19763358_


\subsection {reciprocal sapce 倒易空间}

考虑任何一个描述晶体性质的方程(比如说电子密度)

由于晶体的周期性,在实空间电子密度是周期性相同的,也就是说对于电子密度$n(r+T(r'))=n(r)$其中T是任何的translation操作

\subsubsection{一维理解}

对于一个一维的,总共三个晶格的晶体,晶格常数为a

---《原子》---a---《原子》---a---《原子》---

那么电子密度n(x)必须满足:

    n(x)    =    n(x+a)       =n(x+2a)  
         
当0<x<a

因为n(x)有这样的周期性,所以一定可以对其进行傅里叶变化

傅里叶变化是从x空间变为q空间$n(x)->n(q)$

对于这个长度为3的一维晶体,其中含有3个cell,那么晶格个数N=3

从x空间到q空间的傅里叶变化是

$n(q)=\frac{1}{\Omega_{crystal}}\int_{\Omega_{crystal}} dxn(x)exp(iq·x)$

代入条件

$n(q)=\frac{1}{3a}\int_0^{3a} dxn(x)e^{(iq·x)}$

对于每一个晶格内部进行积分

$n(q)=\frac{1}{3a}(\int_0^adxn(x)e^{iq·x} +\int_a^{2a}dxn(x)e^{iq·x}+\int_{2a}^{3a}dxn(x)e^{iq·x})$

$n(q)=\frac{1}{3a}(\int_0^adxn(x)e^{iq·x} +\int_0^{a}dxn(x+a)e^{iq·(x+a)}+\int_{0}^{a}dxn(x+2a)e^{iq·x+2a})$

由于n(x)周期性n(x)=n(x+a)=n(x+2a)

$n(q)=\frac{1}{3a}(\int_0^adxn(x)e^{iq·x} +\int_0^{a}dxn(x)e^{iq·(x+a)}+\int_{0}^{a}dxn(x)e^{iq·x+2a})$

把exp拆开

$n(q)=\frac{1}{3a}(\int_0^adxn(x)e^{iq·x} +\int_0^{a}dxn(x)e^{iq·x}e^{iq·a}+\int_{0}^{a}dxn(x)e^{iq·x}e^{iq·2a})$

合并得到

$n(q)=\frac{1}{3a}(e^{iq·0a}+e^{iq·a}+e^{iq·2a})(\int_0^adxn(x)e^{iq·x})$

从而得到三维大晶体的一般形式:

$f(q)=\frac{1}{\Omega_{crystal}}\sum_0^{N_xa1}\sum_0^{N_ya2} \sum_0^{N_za3}  e^{iq·T(n1,n2,n3)} \int_0^{a1} \int_0^{a2} \int_0^{a3} dxdydzf(x,y,z) e^{iq·(x,y,z)}$

$f(q)=\frac{1}{\Omega_{crystal}}\sum e^{iq·T} \int drf(r) e^{iq·r}$

T是所有可能的平移(translation)操作

\subsubsection {q需要满足的条件}

根据Born-Von Karmen条件,傅里叶变化后的每一个component都必须满足 $exp(i q·Na)=1$

为了满足$exp(iq·3a) =1$,则必须有:

$q·a=2\pi \frac{integer}{3}$

对于大晶体,$q·a=2\pi \frac{integer}{N}$

N是一个很大的值,所以大晶体的q点是连续的

\subsubsection {倒易矢量的定义}

继续对于$q·a=2\pi \frac{integer}{3}$

尝试当interger=1

$q·a=\frac{2\pi}{3}$

$q=\frac{2\pi}{3a}$

$n(q)=\frac{1}{3a}(e^{0}+e^{\frac{i2\pi}{3}}+e^{\frac{i4\pi}{3}})(\int_0^adxn(x)e^{iq·x})$

然而

$e^{0}+e^{\frac{i2\pi}{3}}+e^{\frac{i4\pi}{3}}=0$

所以$n(\frac{2\pi}{3a})=0$

之后发现只有当$q·a=2\pi · integer$时(而非$q·a=2\pi \frac{integer}{3}$)

n(q)才有

$n(q)=\frac{1}{3a}(e^{2\pi}+e^{2\pi}+e^{2\pi})(\int_0^adxn(x)e^{2\pi})$

$n(q)=\frac{1}{3a}(1+1+1)(\int_0^adxn(x)*1)$

也就是说对任意integer

$n(\frac{2\pi · integer}{a})=\frac{1}{a}(\int_0^adxn(x))$

可以说,q具有周期,周期是$q=\frac{2\pi · integer}{a}$



从而定义单位倒易矢量$b=\frac{2\pi}{a}$

注意$n(q)=n(q+\frac{2\pi}{a})$这就是为什么只需要考虑第一布里渊区


使n(r)的fourier component 非0的q的集
合记为G(倒易空间),

$G(m)=m * b$
 m为任意正整数

m=0 为基态

**推广到三维大晶体**


$n(x+x',y+y',z+z')=n(x,y,z)$,其中x' y' z'是整数倍的$a_1 a_2 a_3$

$a_1 a_2 a_3$是三个晶格向量

傅里叶变化可以对所有有周期性性质的函数使用

因为n(r)有这样的周期性,所以一定可以对其进行傅里叶变化

傅里叶变化是$n(r)->n(q)$

对于一个体积为$\Omega$的三维晶体,其中含有$N_x \times N_y \times N_z = N_{cell}$个cell的体系 

根据Born-Von Karmen条件,傅里叶变化后的每一个component都要满足 $exp(iq· N_x a_1)=exp(iq· N_ya_2)=exp(iq· N_z a_3)=1$



为了满足$exp(iq· N_x a_1)=exp(iq· N_ya_2)=exp(iq· N_z a_3)=1$,则必须有:

$q·(a_1,a_2,a_3)=2\pi \frac{integer}{N_i}$

从r到q的傅里叶变化是

$f(q)=\frac{1}{N_{cell}\Omega_{cell}}\sum e^{iq·T} \int drf(r) e^{iq·r}$

通过计算,q需要满足的条件从
$q·(a_1,a_2,a_3)=2\pi \frac{integer}{N_i}$

变为

$q·(a_1,a_2,a_3)=2\pi · integer$

对于满足条件的q的集合记为G,
$G(m_1,m_2,m_3)=m_1b_1,m_2b_2,m_3b_3$

其中$m_1,m_2,m_3$是任意整数,$b_1,b_2,b_3$是$a_1 a_2 a_3$的倒向量

\subsection {第一布里渊区}

第一布里渊区是倒空间的wigner-seitz cell

同样用上面的3a大小的一维晶体距离

注意倒空间是无限长的一维空间,倒空间两个相邻格点之间距离为$b=\frac{2\pi}{a}$

第一布里渊区是

$-\frac{\pi}{a}<q<+\frac{\pi}{a}$

q=0对应了基态



\subsection {reference book 参考书}

David S. Sholl, _Density functional Theory_

Kittel, _solid state physics_

Dehoff _Thermodynamics_


\end{document}